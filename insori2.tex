\documentclass[deutsch]{lib/llncs/llncs}
\usepackage{lib/llncs/llncsdoc}
\usepackage[ngerman]{babel}
\usepackage[utf8]{inputenc}
\usepackage{hyperref}
\usepackage{graphicx}
\usepackage{lib/picins/picins}
\usepackage[nottoc]{tocbibind}


\begin{document}
\markboth
{Anwendung des ''Technology Acceptance Model'' zur Akzeptanzbestimmung qualifizierter elektronischer  Fernsignaturen im Unternehmensumfeld}
{Anwendung des ''Technology Acceptance Model'' zur Akzeptanzbestimmung qualifizierter elektronischer  Fernsignaturen im Unternehmensumfeld}
\thispagestyle{empty}


\begin{flushleft}
\LARGE\bfseries Anwendung des ''Technology Acceptance Model'' zur Akzeptanzbestimmung qualifizierter elektronischer  Fernsignaturen im Unternehmensumfeld


\end{flushleft}
\rule{\textwidth}{1pt}
\vspace{2pt}


\begin{flushright}
\Huge


\begin{tabular}{@{}l}
Interdisziplinäre und \\
sozialwissenschaftliche \\
Reflexion der Informatik 2\\\\
Wintersemester 2017/2018\\\\
Frank Dreyer\\
Matrikelnummer: 741827\\\\
07.02.2018\\[6pt]
\end{tabular}


\end{flushright}
\rule{\textwidth}{1pt}
\vfill

\newpage
\tableofcontents
\newpage\vspace{2pt}


\section{Abstract}
Genehmigungen begleiten Unternehmen weltweit beim täglichen Arbeitsalltag, sei es der Antrag auf Urlaub oder die Genehmigung eines Projekts beim Kunden. All diese Einwilligungen werden momentan noch vorrangig handschriftlich unterzeichnet, was nicht nur mit einem hohen Archivierungsaufwand verbunden ist, sondern auch die Kosten eines Unternehmens in die Höhe treiben kann. Experten schätzen, dass jede Transaktion, die Papier, Drucken, Unterschreiben, Scannen, Versand, Archivierung und Ersatz verlorengegangener Dokumente einschließt, 30 Euro für ein Unternehmen verursacht. \\
Eine Alternative zur handschriftlichen Unterschrift bilden 'Digitale Signaturen'. Diese Signaturverfahren arbeiten vollkommen papierfrei und sichern anhand eines asymmetrischen kryptographischen Verfahren neben der Authentizität eines Dokuments auch die Datenintegrität ab, was ein hohes Maß an Sicherheit garantiert. \\
Trotz all dieser Vorteile werden Digitale Signaturen im Geschäftsumfeld bisher wenig eingesetzt. Eine neue Verordnung der EU, die 'Verordnung über elektronische Identifizierung und Vertrauensdienste für elektronische Transaktionen im Binnenmarkt' (eIDAS), welche 2016 in Kraft getreten ist, verspricht Besserung. \\
Diese Arbeit zeigt, warum Digitale Signaturen im Geschäftsumfeld bisher wenig Akzeptanz gefunden haben und warum durch die eIDAS-Verordnung geltenden 'Qualifizierten Elektronischen Fernsignaturen' eine breite Akzeptanz bei Unternehmen finden werden. \\
Um die Akzeptanz in diesem Kontext zu evaluieren, wird das \textit{Technology Acceptance Model} angewandt.


\newpage


\section{Grundlagen}


\subsection{Technology Acceptance Model}
Das \textit{Technology Acceptance Model} \cite{Zitat01}, kurz TAM, ist ein von von Fred D. Davis entwickeltes Akzeptanzmodell, welches auf dem sozialpsychologischem Modell \textit{Theory of Reasond Action} (TRA) von Ajzen und Fishbein \cite{Zitat04} basiert. Das Modell zielt darauf ab, Erkenntnisse darüber zu gewinnen, ob und warum Personen Technologien akzeptieren oder diese ablehnen. \\
Dabei wird angenommen, dass eine Person mit positiver Nutzungseinstellung zur Technologie diese auch tatsächlich verwendet. (Vgl. \cite[S. 237]{Zitat03}) Diese Nutzungseinstellung hängt wiederum maßgeblich von den Faktoren 'wahrgenommener Nutzen' und 'wahrgenommener Bedienungskomfort' ab. \\
Der 'wahrgenommene Nutzen' beschreibt das subjektiv Empfinden, dass sich eine Technologie positiv auf die Steigerung der eigenen Arbeitsleistung in einem organisatorischem Kontext auswirkt.  (Vgl. \cite[S. 320]{Zitat02}) \\
Der 'wahrgenommene Bedienungskomfort' bezeichnet das subjektive Empfinden, dass die Verwendung einer Technologie mit wenig Aufwand verbunden ist, bzw. dass die Technologie einfach zu benutzen ist. (Vgl. \cite[S. 320]{Zitat02}) \\
Bandow und Holzmüller fassen die Auswirkung dieser beiden Determinanten folgendermaßen zusammen: ''Je größer der Nutzen eines Informationssystems und je einfacher dessen Bedienbarkeit, desto eher ist der Anwender dazu bereit, das neue System zu nutzen.'' (Vgl. \cite[S. 237]{Zitat03}) \\
Auf den 'wahrgenommenen Nutzen' wie auch den 'wahrgenommenen Bedienungskomfort' wirken wiederum 'externe Variablen', die unter anderem demografische und persönliche Merkmale des Akteurs umfassen, im Originalmodell allerdings nicht weiter spezifiziert werden. (Vgl. \cite[S. 21]{Zitat01}) \\
Abbildung 1 illustriert das Modell.
\begin{figure}
	\centering
	\includegraphics[scale=0.40]{img/abbildung1.png}
	\caption{\textit{Technology Acceptance Model} (Vgl. \cite[S. 237]{Zitat03})}
\end{figure}


\subsection{Digitale Signaturen}
Digitale Signaturen sind in Grenzen vergleichbar mit konventionellen Unterschriften auf Papier, welche über ein asymmetrisches kryptographisches Verfahren digitalisiert wurden. (Vgl. \cite[S. 297]{Zitat06}) Dieses asymmetrische Kryptosystem schließt drei Prozesse mit ein: den 'Unterschriftsprozess', den 'Authentifizierungsprozess' sowie den 'Prozess zur Sicherung der Datenintegrität'. (Vgl. \cite[S. 4]{Zitat05}) \\
Beim 'Unterschriftsprozess', muss sich die Person, die das jeweilige Dokument unterzeichnen soll, zunächst identifizieren. Die Identifikation hängt von der Implementierung ab und kann dementsprechend unter anderem über eine PIN, ein Passwort oder auch einen sequenz-basierten Token-Code abgewickelt werden. Ist die Identität des Unterzeichners bestätigt, so erhält er ein Zertifikat mit seiner Identität und einem öffentlichem und privatem Schlüsselpaar, mit dem er das Dokument signieren kann. Dafür wird zunächst ein einzigartiger mathematischer Code aus dem Dokument generiert, der im Anschluss über den privaten Schlüssel verschlüsselt, bzw. ''signiert'', wird. Um den 'Unterschriftsprozess' abzuschließen wird das Dokument zusammen mit der Signatur (dem verschlüsseltem mathematischen Code) an den Empfänger gesandt. (Vgl. \cite[S. 4]{Zitat05}) \\
Der Empfänger kann daraufhin im 'Authentifizierungsprozess' überprüfen, ob das Dokument auch von der richtigen Person unterzeichnet wurde. Dafür fordert er den öffentlichen Schlüssel der Unterzeichners an und kann daraufhin mit dem öffentlichen Schlüssel den mathematischen Code entschlüsseln. Das Entschlüsseln wird nur dann funktionieren, wenn das Dokument mit dem korrespondierenden privaten Schlüssel verschlüsselt wurde und kann daraus folgernd auch die Authentizität garantieren. (Vgl. \cite[S. 4]{Zitat05}) \\
Zum Schluss wird im 'Prozess zur Sicherung der Datenintegrität' überprüft, ob das Dokument nach dem Signieren nicht mehr verändert wurde. Dafür wird erneut der mathematische Code aus dem Dokument generiert und mit dem im 'Authentifizierungsprozess' entschlüsselten mathematischen Code verglichen. Beide müssen identisch sein um sicher zu sein, dass das Dokument nach dem Unterschreiben nicht modifiziert wurde.  (Vgl. \cite[S. 4]{Zitat05})


\subsection{Qualifizierte Elektronische Signaturen}
Von Qualifizierten Elektronischen Signaturen spricht man, wenn die Erzeugung einer digitalen Signatur dezentral über eine sichere Signaturerstellungseinheiten abgewickelt wird und die Signatur auf qualifizierten Zertifikaten beruht. (Vgl. \cite[S. 8]{Zitat07}) \\
Die Signaturstellungseinheit, die auch Zertifizierungsstelle, Zertifizierungsdienstleister, \textit{Certification Authority} (CA) oder auch \textit{Public Key Infrastructure} (PKI) genannt wird, muss dabei vom Unterzeichner, als auch allen, die sich auf die Signatur verlassen oder berufen wollen, vertraut werden. (Vgl. \cite[S. 9]{Zitat07}) \\
Darüber hinaus sind Zertifizierungsstellen, wie der Name bereits andeutet, für die Vergabe qualifizierter Zertifikate verantwortlich. 
Diese Zertifikate müssen bestimmte Angaben, wie z.B. den Namen des Inhabers, Angaben des Signaturschlüssels und den Gültigkeitszeitraum des Zertifikats enthalten. Außerdem müssen vor der Erstellung eines Zertifikats die Identität des zukünftigen Inhabers anhand von Ausweispapieren geprüft werden und der Zertifizierungsdienstleister muss die Zertifikate über einen Zeitraum von mindestens fünf Jahren über offene Kommunikationskanäle für jedermann öffentlich zugänglich machen. (Vgl. \cite[S. 9]{Zitat07}) 


\subsection{Qualifizierte Elektronische Fernsignaturen}
Von Qualifizieren Elektronischen Fernsignaturen spricht man, wenn bei einer Qualifizierten Elektronischen Signatur der private Schlüssel des Zertifikatinhabers bei der Signaturstellungseinheit liegt und von ihr verwaltet wird, wodurch der Signierer keine Smartcards, bzw. Speichermedien mit dem zugehörigen privaten Schlüssel, bei sich tragen muss. (Vgl. \cite[S. 30]{Zitat08})


\section{Der wahrgenommene Nutzen Qualifizierter Elektronischer Fernsignaturen}


\subsection{Effizienzsteigerung und Kostenminimierung dank schlanker Prozesse}
Geschäftsabschlüsse, die mit einer handgeschriebene Unterschrift auf Papier durchgeführt werden, haben zwei entscheidende Nachteile: Sie verzögern die Wirtschaftlichkeit und steigern die Komplexität durch Archivierung. (Vgl. \cite[S. 3]{Zitat05})\\
Diese Nachteile können mit digitalen Signaturen, die über eine Zertifizierungsstelle durchgeführt und verwaltet werden, vermieden werden. Genehmigungen, die bisher durch Prozesse wie Drucken, Versand, Scannen und Archivierung viel Zeit in Anspruch und Kosten durch Papier und Personal verursacht haben, können dadurch praktisch ohne Zeitverzögerung kostengünstig abgewickelt werden und sind Dank der Verwaltung durch die Signaturstellungseinheit leicht und jeder Zeit einsehbar. \\
Dieser Sachverhalt wird auch in einer von Arthur D. Little durchgeführten Befragung von 50 Marktexperten bestätigt. Sie schreiben: ''Viele Unternehmen und Behörden haben bereits das Potential dieser Technologie erkannt. Ein vollständig digitaler Prozess für die Signierung und den Versand von Dokumenten senkt die Arbeitszeiten und Kosten für Papier und Transport.'' (Vgl. \cite[S. 7]{Zitat05})
Darüber hinaus nennen die Teilnehmer der Befragung die Kosten- wie auch die Zeitersparnis als die zwei wichtigsten Vorteile beim Einsatz von Digitalen Signaturen. (Vgl. Abbildung \cite[S. 7]{Zitat05})
Zwar könnte man argumentieren, dass Digitale Signaturen aufgrund der Implementierung hohe Kosten verursachen. Allerdings werden diese entstandenen Kosten schnell durch reduzierte Kosten aufgewogen. (Vgl. \cite[S. 7]{Zitat05})


\subsection{Rechtsgültige Sicherheitsgarantie durch asymmetrisches Kryptosystem}
Damit eine Willenserklärung verbindlich und vor Gericht gültig ist, muss eine Authentizität bzw. Nichtabstreitbarkeit gewährleistet sein. In Schriftform wird dies durch eine eigenhändige Namensunterschrift erreicht. (Vgl \cite[S. 5-6]{Zitat07}) Diese Art der Unterschrift birgt allerdings einige Gefahren in Bezug auf die Sicherheit. So können von Hand gefertigte Unterschriften leicht gefälscht werden von Personen, die sich als Willenserklärende ausgeben. Darüber hinaus können unterschriebene Dokumente im Nachhinein von Unberechtigten modifiziert werden. Zwar werden solche Fälle der Urkundenfälschung in den meisten Fällen aufgeklärt, allerdings zeigt die polizeiliche Kriminalstatistik des Bundeskriminalamts aus dem Jahr 2016, dass es mit einem verhältnismäßig hohen Anteil von 16,4 \% immer noch Urkundenfälschungsdelikte gibt, die nicht aufgeklärt werden können. (Vgl. \cite[S. 34]{Zitat10})\\
Qualifizierte elektronische Fernsignaturen verwenden eine ganz andere Taktik. Sie benutzen ein asymmetrisches kryptographisches Verfahren um die Authentizität zu gewährleisten. Dafür muss sich die für die Unterschrift berechtigte Person zunächst mittels einer 2-Faktor-Authentifizierung bei der Signaturstellungseinheit identifizieren, die bei Erfolg den Prozess des Signierens für den identifizierten Zertifikatinhaber einleitet. Damit die Signaturstellungseinheit als ''qualifizierter Vertrauensdienst'' (Vgl. \cite[S. 30]{Zitat08}) operieren darf, müssen gemäß der EU-Verordnung eIDAS gewisse Sicherheitskriterien erfüllt sein: So müssen ''Anbieter von elektronischen Fernsignaturdiensten spezielle Verfahren für die Handhabung und Sicherheitsverwahrung mit vertrauenswürdigen Systemen und Produkten anwenden, u. a. durch abgesicherte elektronische Kommunikationskanäle, um für eine vertrauenswürdige Umgebung zur Erstellung elektronischer Signaturen zu sorgen und zu gewährleisten, dass diese Umgebung unter alleiniger Kontrolle des Unterzeichners genutzt worden ist.'' (Vgl. \cite[S. 233]{Zitat09}) Des Weiteren muss sich der qualifizierte Vertrauensdienst einer Konformitätsprüfung unterziehen, die alle 24 Monate wiederholt wird und überprüft ob alle in der eIDAS-Verordnung festgelegten Anforderungen erfüllt sind. (Vgl. \cite[S. 30]{Zitat08}) \\
Ein weiterer wichtiger Mehrwert von digitalen Signaturen, der bei herkömmlichen eigenhändigen Namensunterschriften fehlt, ist die Integritätsprüfung, die die Manipulation eines Dokuments praktisch unmöglich macht. (Vgl. \cite[S. 7]{Zitat05}) \\
Eine Kombination der beiden Sicherheitsfaktoren Authentizität und Integrität ist gerade bei Geschäftstransaktionen entscheidend, da in diesem Umfeld mit sensitiven und vertraulichen Daten gearbeitet wird. (Vgl. \cite[S. 7]{Zitat05})


\subsection{Weitere Besonderheiten digitalisierter Willenserklärungen}
Neben der effizienzsteigernden, kostenminimierenden und sicherheitsgarantierenden Eigenschaften, die qualifizierte elektronische Signaturen mit sich bringen, haben sie eine Reihe weiterer Besonderheiten, die vor allem bei Transaktionen im Geschäftsumfeld den Unterschied machen. \\
Unternehmen die sich bei ihren Geschäftspartnern und Kunden fortschrittlich repräsentieren wollen, haben mit Digitalen Signaturen ein hervorragendes Werkzeug dafür. Sie vertreten damit ein innovatives Image sowohl für Nutzer als auch für Anbieter der Technologie, was sich im Rückschluss auf eine positive Kundenzufriedenheit auswirkt. (Vgl. \cite[S. 7]{Zitat05}) Dieses Image spiegelt sich auch im B2B-Umfeld wieder, wo ''technologisch fortgeschrittene Geschäftspartner als Rollenmodelle dienen.'' (Vgl. \cite[S. 12]{Zitat05}) \\
Ein weiterer Vorteil ist der Zeitstempel, welcher zu einer Digitalen Signatur hinzugefügt werden kann. So ist es möglich sicher zu stellen, dass ein Dokument zu einem bestimmten Datum signiert wurde. (Vg. \cite[S. 7]{Zitat05}) Ist ein Dokument zu einem bestimmten Termin zu unterschreiben, z.B. bei einer Kündigung, kann dieser exakte bestimmte Zeitstempel der Unterschrift helfen Betrug auszuschließen. \\
Die eIDAS-Verordnung, welche 2016 in Kraft getreten ist, macht es des weiteren neuerdings möglich so genannte 'elektronische Siegel' oder 'E-Siegel' einzusetzen. Jene sind qualifizierte Signaturen, die nicht einer Person sondern einer Organisation zugeordnet werden, was insbesondere für Unternehmen interessant werden könnte. Geschäftseinwilligungen, die einen Firmenstempel erfordern, können dementsprechend auch digitalisiert werden und folglich Prozesskosten minimieren. (Vgl. \cite[S. 30-31]{Zitat05})


\subsection{Zwischenfazit: Wird der Nutzen von Qualifizierten Elektronischen Fernsignaturen wahrgenommen?}
Diese Frage ist mit einem eindeutigen Ja zu beantworten. Qualifizierte elektronische Fernsignaturen lassen Unternehmen schneller agieren bei Geschäftsabschlüssen, sind sicherer und kostengünstiger als herkömmliche von Hand gefertigte Namensunterschriften und bieten weitere nützliche Features. \\
Firmen haben durch den Einsatz besagter Technologie das Potential nachhaltige Wettbewerbsvorteile zu erzielen, wie eine Untersuchung von Arthur D. Little und weitere Analysen bestätigen. (Vgl. \cite[S. 7]{Zitat05})
Dieses Potential wurde bereits von zahlreichen Firmen erkannt. So auch bei der \textit{Credit AgricoleConsumer Finance}, einem Anbieter von Konsumentenkrediten in Frankreich und Europa, der durch den Einsatz die Kundenzufriedenheit als innovativer Anbieter erhöhen sowie Kosten sparen konnte. \cite[S. 13]{Zitat05} \\
Zusammenfassend kann festgestellt werden, dass sich qualifizierte elektronische Fernsignaturen positiv auf die Steigerung der eigenen Arbeitsleistung auswirkt, da nachhaltige Wettbewerbsvorteile erzielt werden können. Der wahrgenommene Nutzen ist demzufolge sehr hoch. 


\section{Der wahrgenommene Bedienungskomfort Qualifizierter Elektronischer Fernsignaturen}


\subsection{Bedienungsproblematik beim bisherigen Einsatz von Qualifizierten Elektronischen Signaturen}
Auch wenn sich der Einsatz qualifizierter elektronischer Signaturen - dank ihrer Eigenschaften - positiv auf die Steigerung der eigenen Arbeitsleistung auswirkt, gab es bisher einige Defizite was die Bedienung betrifft. \\
Dieser Mangel an Bedienungskomfort ist vor allem durch die unterschiedlich ausfallenden Signaturgesetzgebungen der einzelnen Länder innerhalb der EU geschuldet. Zwar gab es bereits eine EU-Richtlinie, die versuchte Einigkeit digitalisierter Willenserklärungen zu schaffen, allerdings legten die EU-Mitglieder Wert auf unterschiedliche Aspekte Digitaler Signaturen, wodurch keine Einigung erzielt werden konnte. \cite[S. 30]{Zitat08} Jene ''babylonische Gesetzesverwirrung'' \cite[S. 30]{Zitat08} hatte zur Folge, dass die Mitglieder aufgrund variierender Sicherheitsmaßstäbe unterschiedliche Technologien für den Einsatz digitalisierter Signaturen wählten, was eine Interoperabilität zwischen den Mitgliedern praktisch unmöglich machte. \\
Deutschland stellte hohe Sicherheitsanforderungen an die Technologie. Qualifizierte elektronische Signaturen waren demzufolge nur gestattet wenn ''der Signierer eine Smartcard (oder ein anderes Speichermedium) mit dem privaten Signaturschlüssel bei sich trug und unter seiner direkten Kontrolle hatte.'' \cite[S. 30]{Zitat08} Qualifizierte Fernsignaturen, bei denen der private Signaturschlüssel bei der Signaturstellungseinheit erstellt und verwaltet wird, waren daher nicht erlaubt. \\
Der Einsatz solcher Smartcards bzw. Tokens erweist sich als äußerst unpraktisch bei der Bedienung: \\
So müssen zusätzliche Geräte angeschafft und Software installiert werden, um die Technologie überhaupt einsetzen zu können, die dann auch noch umständlich in der Bedienung ist. Diese Hürde für den Endanwender wird auch von Diplom-Betriebswirt Enrico Entschew bestätigt, der als Senior Developer für die Bundesdruckerei tätig ist: ''Je komplexer die Einsatzumgebung gestaltet ist, desto höher ist die Barriere für den Nutzer.'' \cite[S. 232]{Zitat09} \\
Ein weiteres wichtiges Problem ergibt sich dadurch, dass Nutzer der Technologie gezwungen sind die für die Signatur notwendigen Geräte immer bei sich zu tragen. Dass das den Bedienkomfort einschränkt und bei Unternehmen schlecht ankommt, spiegelt sich auch in einem Interview wieder, das Arthur D. Little durchgeführt hat. So sagt ein Studienteilnehmer: ''Warum sollten wir von unseren Kunden verlangen, ständig einen Token oder Smart-Card-Reader bei sich zu tragen?'' \cite[S. 5]{Zitat05} \\
All diese Komfortdefizite in der Bedienung haben letztendlich dazu geführt, dass qualifizierte elektronische Signaturen bisher eher als technische Spielerei anstatt einer ernsthaften Innovation angesehen wurden.  \cite[S. 29]{Zitat08}


\subsection{Besserung durch eIDAS}


\subsection{Zwischenfazit: Wird der Bedienungskomfort von Qualifizierten Elektronischen Fernsignaturen wahrgenommen?}


\section{Fazit}


\bibliographystyle{amsalpha}
\bibliography{lit/lit}


\end{document}
